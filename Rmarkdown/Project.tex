\documentclass[12pt,]{article}
\usepackage{lmodern}
\usepackage{amssymb,amsmath}
\usepackage{ifxetex,ifluatex}
\usepackage{fixltx2e} % provides \textsubscript
\ifnum 0\ifxetex 1\fi\ifluatex 1\fi=0 % if pdftex
  \usepackage[T1]{fontenc}
  \usepackage[utf8]{inputenc}
\else % if luatex or xelatex
  \ifxetex
    \usepackage{mathspec}
  \else
    \usepackage{fontspec}
  \fi
  \defaultfontfeatures{Ligatures=TeX,Scale=MatchLowercase}
\fi
% use upquote if available, for straight quotes in verbatim environments
\IfFileExists{upquote.sty}{\usepackage{upquote}}{}
% use microtype if available
\IfFileExists{microtype.sty}{%
\usepackage{microtype}
\UseMicrotypeSet[protrusion]{basicmath} % disable protrusion for tt fonts
}{}
\usepackage[margin=0.75in]{geometry}
\usepackage{hyperref}
\PassOptionsToPackage{usenames,dvipsnames}{color} % color is loaded by hyperref
\hypersetup{unicode=true,
            pdfauthor={R.L., L.Z.},
            colorlinks=true,
            linkcolor=red,
            citecolor=red,
            urlcolor=blue,
            breaklinks=true}
\urlstyle{same}  % don't use monospace font for urls
\usepackage{graphicx,grffile}
\makeatletter
\def\maxwidth{\ifdim\Gin@nat@width>\linewidth\linewidth\else\Gin@nat@width\fi}
\def\maxheight{\ifdim\Gin@nat@height>\textheight\textheight\else\Gin@nat@height\fi}
\makeatother
% Scale images if necessary, so that they will not overflow the page
% margins by default, and it is still possible to overwrite the defaults
% using explicit options in \includegraphics[width, height, ...]{}
\setkeys{Gin}{width=\maxwidth,height=\maxheight,keepaspectratio}
\IfFileExists{parskip.sty}{%
\usepackage{parskip}
}{% else
\setlength{\parindent}{0pt}
\setlength{\parskip}{6pt plus 2pt minus 1pt}
}
\setlength{\emergencystretch}{3em}  % prevent overfull lines
\providecommand{\tightlist}{%
  \setlength{\itemsep}{0pt}\setlength{\parskip}{0pt}}
\setcounter{secnumdepth}{5}
% Redefines (sub)paragraphs to behave more like sections
\ifx\paragraph\undefined\else
\let\oldparagraph\paragraph
\renewcommand{\paragraph}[1]{\oldparagraph{#1}\mbox{}}
\fi
\ifx\subparagraph\undefined\else
\let\oldsubparagraph\subparagraph
\renewcommand{\subparagraph}[1]{\oldsubparagraph{#1}\mbox{}}
\fi

%%% Use protect on footnotes to avoid problems with footnotes in titles
\let\rmarkdownfootnote\footnote%
\def\footnote{\protect\rmarkdownfootnote}

%%% Change title format to be more compact
\usepackage{titling}

% Create subtitle command for use in maketitle
\newcommand{\subtitle}[1]{
  \posttitle{
    \begin{center}\large#1\end{center}
    }
}

\setlength{\droptitle}{-2em}
  \title{\Large{\textbf{Player Performance, Social Power and Team Valuation in 2016-2017 NBA Season Project}}}
  \pretitle{\vspace{\droptitle}\centering\huge}
  \posttitle{\par}
  \author{R.L, L.Z.}
  \preauthor{\centering\large\emph}
  \postauthor{\par}
  \predate{\centering\large\emph}
  \postdate{\par}
  \date{STAT 6950}


\begin{document}
\maketitle

\section{Introduction}\label{intro}

National Basketball Association(NBA) is no doubt one of the most
successful sport league in the world. The successful commercialisation
of NBA makes it possible for teams to offer high salaries to players,
which encourages players to play harder for their future contracts.
Besides, elite players who perform impressively usually have large
numbers of followers on twitter, or are searched frequently on websites,
which means they might have large social powers. In view of these,
valuations of sports teams, whose main assets are players and especially
the elite players, might be influenced by players' performance and their
social powers. So for team managers, how can they determine suitable
salary for players and what strategies can they take to improve team
valuation?

Intuitively speaking, the salaries of players are mainly determined by
their performance. But considering there are so many variables to
evaluate players' performance, managers might wonder which groups of
them are most important. In addition, lots of players have their own
social media accounts, which is a good strategy to expose themself to
public. We might have good reasons to cast doubts on whether the power
of social media will have some impacts on their salary. All of these
will be investigated and explained in the main body, which could be used
to answer the question: what determines players' salaries in NBA league?

Another issue managers have to deal with is the team valuation. The
valuation of a sport team is a bit hard to determine in corporation
finance. Generally speaking, the income from tickets, the values of
brands, the operation expenses are three main factors that make up for
team valuations. These factors are associated with team performance,
marketing strategy, salary structure, etc, many of which might be
covered in the determimants of salaries as well. Given these
information, the managers might still feel confused at what on earth
determines the team valuation. Statistical analysis is provided to
eliminate confusions and the managers can know what to do to increase
the team valuation.

Our project aims to answer both questions raised above by using
appropriate statistical models. Lending from these models, we hope to
provide business insights for team managers to improve their teams in
multi-dimensions. \section{Dataset} With the help of background
information above, we find relevant variables from multiple sources. A
main part of our dataset comes from Kaggle
(\url{https://www.kaggle.com/noahgift/social-power-nba}). In this part,
data about the team valuation, average attendence of fans per game,
performace statistics, salaries and social power of elite players in a
team are provided. Besides, we also find regional GDP data from Bureau
of Economic Analysis (\url{https://www.bea.gov/}) and 2016-17 NBA ticket
prices from VIVIDSEATs
(\url{https://www.vividseats.com/blog/nba-ticket-prices}). All of these
make up for our dataset for analysis.

Some interesting issues occur in our exploratory data analysis. Firstly,
players in different positions are good at different skills. For
example, players in position center tend to have better performances in
rebound statistics, which makes sense since they are usually higher then
players in other position on average. More importantly, this informs us
the importance of distinguish players in different position in the
salary model since players have different strengths. Besides, we also
depict the scatter plot matrix for relevant statistcis about player
performances and social power, team performance and social power, etc. A
good suggestion in these scatter plot matrix is we should do the
logarithm transformation when investigate the team valuation. We also
need to delete some outliers/influential points to regularize the
dataset. All these plots are present in Appendix \ref{EDA}.

\section{Player Salary: Method and Results}

Since we are exploring the relationship between players' salary and
their statistics, we will use players' salary as response variables, and
their statistics as explantory variables, to build a multiple linear
regression model:

\begin{equation}\label{MLR}
Y_{i} = \beta_0+\beta_1X_{1,i}+\cdots+\beta_{p-1}X_{p-1,i} + \varepsilon_i,~~i=1,2,\cdots,n.
\end{equation}

where \(\varepsilon_i~iid\sim Normal(0,\sigma^2)\). Here, the \(Y_i\)
represents the salary of the \(i\)th player, and \(X_{i,j}\) represents
the \(j\)th statistic of the \(i\)th player. Because we do not have any
prior knowledge in basketball, we decided to consider all variables into
our model at first in order to see whether these explantory variables
are useful to explain our response varialbe. The \(R^2\) of the full
model is 0.7 (we omit the summary.lm here because it is too long), which
is a relatively high number. That means our explantory variables can be
helpful to explain about 70\% variability of the response data around
its mean. Although this full regression model is useful in predicting
the player's salary, our goal is to provide an simple and interpretable
model to help team manager make a decision. Therefore, to simplfy our
model, we will do the variable selection at first and choose explantory
variables as few as possible.

\subsection{Potential Predictors} \label{PP}

The dataset we collected contains many different type of potential
predictors (explanatory variables). Some of these are continuous
measurements, like the ``Field Goals'' or ``Free Throws'' of a player.
Some of them are discrete but ordered, like a player's age. Others can
be categorical, like the position of a player. All these types of
potential predictors can be useful in our multiple linear regression
(\ref{MLR}). The potential predictors we consider are:

\begin{itemize}
\item \textbf{The intercept}: Suppose we define $\mathbf{1}$ be a predictors that is always equal to 1, then our mean function of (\ref{MLR}) can be written as 
$$ E(Y_{i}|\mathbf{X}) = \beta_0\mathbf{1}+\beta_1X_{1,i}+\cdots+\beta_{p-1}X_{p-1,i},~~i=1,2,\cdots,n.$$
\item \textbf{Continuous measurements and age}: All continuous measurement are included in our regression model without any transformation. We also include the "AGE" without any transformation because we believe it is approximately continuous.

\item \textbf{Dummy variables and factors}: The "POSITION" with 4 levels are considered as \textit{factor}, and they are included in \ref{MLR} using \textit{dummy vaiables}.

\item \textbf{Interactions}: Based on our experience, we think that the "POSITION" of a player will interact with other potential predictors when they influence the salary of this player. For example, "POINTS" is more important to the point guard (PG), while "Defensive Rebounds" is more important to the center (C). Therefore, we include all interaction terms between "POSITION" and other predictors as our potential predictors.
\end{itemize}

To simplify our model and make it interpretable, we do not consider the
transformations or the polynomials of our predictors. We would like to
consider them as our future jobs.

\subsection{Variable Selection}

In our last subsection, we have identified the potential predictors in
our model, but the question for us is how to select the useful variables
and why we need to do the selection. Firstly, we need to interpret our
model to a team manager, so we have to make the model simple. Too many
predictors will definitely increase the complexity of our model. Also,
the collinearity between variables can make our model unreasonable. For
example, in our full model, which contains all potential predictors, the
coefficient of ``POINTS'' is negative, which means the player will get
fewer salary if he get more points in the game. That is obviously wrong,
so we must do the variable selection to let our model seem reasonable.
What's more, we have fewer than 250 data in our dataset, but the
potential predictors are more than 70. Therefore, variable selection can
enhance generalization by reducing overfitting, so it can make more
precise prediction.

The approach to finding useful predictors we use here is considering all
potential predictors, and then select one subset that optimizaes the
criterion (we consider AIC and BIC in this section). We have many
different methods to deal with it. For example, the ``All Subsets''
method and Stepwise Search Methods are both very useful. However,
because we have more than 70 potential explantory variables in our
model, the running time of ``All Subsets'' method will be extremely
long, so it is more sensible for us to use the Stepwise Search Methods
here.

At first, we employ the ``Backward Elimination'', ``Forward Selection''
and ``Stepwise Regression'' to minimize the AIC criterion. We run these
three method, and our results are as follows:

\begin{enumerate}
\item \textbf{Backward Elimination} It chooses almost all potential predictors, and it gets AIC=641.89.
\item \textbf{Forward Selection} It chooses "POINTS + AGE + DRB + PF + AST + eFG. + W + FG. + TOV + ORPM + FGA + PAGEVIEWS + TWITTER FAVORITE COUNT + TWITTER RETWEET COUNT + X2P" as predictor, and it gets AIC=671.14.
\item \textbf{Stepwise Regression} It chooses "POINTS + AGE + DRB + PF + AST + eFG. + W + TOV + ORPM + FGA + PAGEVIEWS + TWITTER FAVORITE COUNT + TWITTER RETWEET COUNT" as predictor and it gets AIC=670.52
\end{enumerate}

Similarly, we employ the BIC criterion and get the following result:

\begin{enumerate}
\item \textbf{Backward Elimination} It chooses ``AGE + MP + FT + MPG + PAGEVIEWS + TWITTER FAVORITE COUNT`` as predictors, and it gets BIC=708.97.
\item \textbf{Forward Selection} It chooses ``POINTS + AGE + DRB + PF`` as predictor, and it gets BIC=711.48.
\item \textbf{Stepwise Regression} It gets the same result as Forward Selection.
\end{enumerate}

All of these three methods choose too many predictors when we use the
AIC criterion. However, when we use BIC criterion, they tend to select
fewer predictors. We think it is because the BIC criterion is largely
influenced by the sample size, so since our dataset is large, it tends
to choose fewer predictors comparing with the AIC. Because we want to
simplify our model and make it interpretable, the predictors we got in
BIC are preferable, and since the ``Forward Selection'' and ``Stepwise
Regression'' get the same result, we will adopt the predictors selected
from them and do the analysis.

\subsection{Diagnostics}

In our multivariate regression model (\ref{MLR}), we have assumed
\(\text{E}(\mathbf{Y}|\mathbf{X})=\mathbf{X\beta}, \text{Var}(\mathbf{Y}|\mathbf{X})=\sigma^2\mathbf{I}\),
and our error terms \(\varepsilon_i\) are
\(iid \sim Normal(0,\sigma^2)\). In order to check the correctness of
our model's assumptions, we analyze the residual
\(\hat{\mathbf{e}} =\mathbf{Y}-\mathbf{\hat{Y}}\). The residuals should
satisfy \(\text{E}(\hat{\mathbf{e}})=\mathbf{0}\) and
\(\text{Var}(\hat{\mathbf{e}})= \sigma^2(\mathbf{I}-\mathbf{H})\), where
\(\mathbf{H}=\mathbf{X'(X'X)^{-1}X}\). We draw the scatterplot of
standardized residual
\(\mathbf{r}=\frac{\mathbf{\hat{e}}}{\hat{\sigma}\sqrt{\mathbf{I-H}}}\)
against the fitted value. Therefore, if our assumptions are correct, the
scatterplot \ref{fig:residual} should not have obvious patterns.

\begin{figure}
\centering
\includegraphics{Proposal_EDA_files/figure-latex/unnamed-chunk-8-1.pdf}
\caption{\label{fig:residual}(1) The first two scatterplots are
standardized residuals against the fitted value before and after the
transformation. (2) The third and forth plots are QQ-plots of
standardized residuals before and after the transformation.}
\end{figure}

It is obvious that our standardized plot (Figure \ref{fig:residual}) has
right-opening megaphone shape, which means our constant variance
assumption has not been satisfied. We can see that
\(\text{Var}(\mathbf Y|\mathbf X)\propto \text{E}(\mathbf Y|\mathbf X)\),
so we do the square root transformation for our response variable
(players' salary) at first, and then repeat our variable selection by
``Stepwise Regression'' and BIC criterion. This time we choose
\texttt{POINTS\ +\ AGE\ +\ DRB} as our response variables and the
BIC=-93.8. We draw a new scatterplot of standardized residual, and it
does not show any pattern this time, which means our constant variance
has been satsified. Also, in the QQ-plot (Figure \ref{fig:residual}), we
can see that most points fit the straight line perfectly after the
transformation. We also perform the Shapiro-Wilk test of normality. The
corresponding p-value is close to 1, which is a perfect support of the
normality assumption of the residuals. Besides, the maximum of Cook's
distance in this model is 0.1, which means no obvious influential point
in our data.

After examining the residuals, checking for influential observations
there is no compelling evidence that any of the usual MLR assumptions
are violated.

\subsection{Cross Validation}

In this subsection, we will adopt the K-fold Cross-Validation method to
assess if our fitted model will be useful for prediction. The idea is to
divide our available data into k equal sized groups. Each time a single
group is retained as the validation data for testing, and the remaining
k-1 groups are used to fit the model. The process will repeat k times,
and k groups used exactly once as the validation data. The so-call
K-fold CV estimator is \[\frac1N\sum_{i=1}^N (Y_i- \hat{Y_i})^2,\] where
\(\hat{Y_i}\) is the predicted value of \(Y_i\) when the model was fit
without the group that case i belongs to. We run the K-fold CV 1 time,
and the K-fold CV estimator is 0.6, which is close to the MSE of our
regression model. The result is shown in the Appendix \ref{KCV}.

\subsection{Conclusion}

After the cross validation, we firmly believe that our model can be
useful in predicting the salary of a basketball player. This
multivariate regression model only contains three predictors,
\texttt{POINTS\ +\ AGE\ +\ DRB} and the intercept, which is easy to
interprete and understand. Based on our result (Appendix \ref{SM}), the
final model is:

\begin{equation}
\sqrt{\text{Salary}} = -1.3582 + 0.0986POINTS + 0.0975AGE + 0.1012DRB
\end{equation}

The interpretation for the coefficient of \texttt{DRB} is, for example,
when Defensive Rebounds increases one unit and other predictors are
fixed, the square root of salary will approximately increase 0.1012
units. From this model, we can see that the coefficients of these
predictors are all positive, which means they have positive contribution
to the salary. We only choose these three predictors does not means
other predictors are not important, instead it means there may exist
strong correlations between these three predictor and other potential
predictors. However, since the \(R^2\approx 0.56\), which means these
three predictors can explain 56\% of variability in salary, which are
sufficient enough for us to do the prediction and provide a useful
suggestion to the team manager.

\section{Team Valuation: Method and Results}

\subsection{Potential Predictors}\label{PP2}

Intuitively, reasonable predictor variables include regional GDP,
tickets prices, and performace statistics, salaries, social power of
elite players. Notice not all performance statitics are reasonable for
our analysis. With the help of information in \ref{intro} part, we can
know only the basic performance statistics seem reasonable since it can
stimulate the purchase of tickets directly. So we only include
\texttt{PTS}, \texttt{rebounds}, \texttt{assists} and some other 5
performance statitiscs variables to reduce the number of variables we
will consider. After that, we use the sum of personal performance
statistics within a team to represent the corresponding team performance
statistics, for example, use the sum of \texttt{PTS} within a team to
represent \texttt{team\_PTS}. Besides, we add up the salaries and social
power variables of players within a team to represent the corresponding
team variables. Lastly, other accessory informaton like
\texttt{TICKET\_PRICE} is also included. Considering the different scale
of the variables, we take the logarithm for some of them. We plot a
summary scatter plot matrix of typical variables below to describe the
relationship between these variables in Appendix \ref{SPM}.

In the scatter plot matrix, we have already delete the strange points in
our dataset, which has been explained in our EDA. There seems to be no
strange behavior after the deletion, which means we can carry on to the
main part of our analysis.

\subsection{Variable Selection}

Given the reasonable variables we analyze in part \ref{PP2}, we employ
backward selection method with BIC criterion to choose the ``best''
linear regression model for analysis. Before implementing the stepwise
search methods, let's first look at the plot of all subsets \ref{BIC2}
to have a basic evaluation of important variables we should pay more
attention. A quick conclusion from this plot is \texttt{lg\_GDP} and
\texttt{lg\_TICKET\_PRICE} are important for BIC selection criterion.
Besides, the inclusion of a social power variables,
\texttt{lg\_TWITTER\_RETWEET\_COUNT} is somewhat surprising since it
might inform us the importance of social power on team valuation. As for
the basic performance statistics, notice only variable \texttt{POINTS}
is excluded, this might be due to the fact that the highest points in
each team will not be so different compared with the other 4 basic
performance statistics.

Now let's use the backward selection method with BIC criterion to
implement model selection, the result is present in \ref{VSR}. The final
result is the same as what the plot above indicates. So we use the
chosen model as the standard model for further analysis, which selects
\texttt{AST}, \texttt{TRB}, \texttt{STL}, \texttt{BLK},
\texttt{lg\_TICKET\_PRICE}, \texttt{lg\_TWITTER\_RETWEET\_COUNT} and
\texttt{lg\_GDP} as predictors.

\subsection{Diagnostics and Analysis}

Considering the small sample we have, We cannot do the lack of fit test
in our problem setting. However, from the diagnostic graph, we can see
there are no severe violations to the regression model, which means the
model we fit is good. So it's safe to use our model selected above. The
fitted result is given below.

Now let's look at the model we use to explore our dataset. One
interesting question is whether team's social power has impacts on
team's valuation. The answer of the question is important since if it is
true, a direct strategy for managers to increase the team valuation is:
sign free players who used to be the all-star. In our model, the only
variable associated with social power is
\texttt{lg\_TWITTER\_RETWEET\_COUNT}, which is siginifcant according to
t test. The result tells us the social power does have impacts on the
team valuation. Also notice the sign of
\texttt{lg\_TWITTER\_RETWEET\_COUNT} is positive, so the managers can
sign the palyers with high \texttt{TWITTER\_RETWEET\_COUNT} to increase
the team valuation. Another interesting result is that variable
\texttt{AST} is not significant in the summary table. A possibile
interpretation for this is: The small ball trend in NBA speeds up the
games, so a team good at blocks, steals will stand out in that trend.
This argument is also supported by the sign of variable \texttt{TRB},
which is negative since the team with more rebounds tends to play more
slowly.

\includegraphics{Proposal_EDA_files/figure-latex/unnamed-chunk-13-1.pdf}

\begin{verbatim}
## 
## Call:
## lm(formula = lg_TEAM_EVALUATION ~ AST + TRB + STL + BLK + lg_TWITTER_RETWEET_COUNT + 
##     lg_GDP + lg_TICKET_PRICE, data = teammat)
## 
## Residuals:
##       Min        1Q    Median        3Q       Max 
## -0.171450 -0.074029  0.005667  0.046353  0.154039 
## 
## Coefficients:
##                          Estimate Std. Error t value Pr(>|t|)    
## (Intercept)               1.18765    0.49924   2.379  0.03108 *  
## AST                      -0.02764    0.01622  -1.704  0.10898    
## TRB                      -0.04159    0.01315  -3.163  0.00643 ** 
## STL                       0.22708    0.10530   2.157  0.04767 *  
## BLK                       0.12958    0.06060   2.138  0.04934 *  
## lg_TWITTER_RETWEET_COUNT  0.23663    0.04443   5.326 8.47e-05 ***
## lg_GDP                    0.29129    0.03562   8.178 6.56e-07 ***
## lg_TICKET_PRICE           0.24447    0.09782   2.499  0.02454 *  
## ---
## Signif. codes:  0 '***' 0.001 '**' 0.01 '*' 0.05 '.' 0.1 ' ' 1
## 
## Residual standard error: 0.1076 on 15 degrees of freedom
## Multiple R-squared:  0.9373, Adjusted R-squared:  0.9081 
## F-statistic: 32.06 on 7 and 15 DF,  p-value: 6.405e-08
\end{verbatim}

\subsection{Conclusion}

Our fitted model suggests a positive association between the counts of
Twitter retweet and Team valuation. A direct conclusion for this is: if
the manager wants to increase the team valuation, one strategy he/she
could take is to sign players in free player market with relatively
large social power. For example, some players who used to be all-stars
but are now in the free player market could be the potential targets.
Reasonable as our result sounds, the value of the coefficient of social
power might be a little confusing, since it indicates 1\% increase in
team's counts of Twitter retweet could increase the team valuation up to
23.6\%. So if we can find more available data and arrange them into a
panel form, we could get a more accurate relationship between the social
power and team valuation.

\section{Discussion}

In this project, we investigate the important factors to determine the
players' salary and team valuation, and we provide our specific answer
and interpretation in each section. Our results show that the models fit
the data pretty well. However, there are still some limitations in our
project. Firstly, we only consider the linear relationship between our
response variables and potential predictors, but we do not consider the
transformations or the polynomials of our potential predictors. So if
there exist any nonlinear relationships between them, our model will
miss these information. Second, in the model selection part, we choose
stepwise regression and BIC criteria to get our final predictors since
it can simplify our model. However, there is no evidence which can
supports our conclusion. To deal with these problems, we can consider
more complex model which includes the transformation of predictors as
potential predictors. Also, asking for some expertise or getting some
prior knowledge can help us choose the correct predictors.

What's more, we also notice that 100 players are considered as elite
players based on their performance in the 2016-2017 season. An
interesting direction of future work is to predict the whether a player
would be a elite player in the following season. Because the simple
variable selection methods are not useful here, we choose the same
predictors as in our multivariate regression model. The preliminary
result is shown in Appendix \ref{LM}, and we would like to investigate
it deeper in the future.

\newpage

\section{Appendix}

\subsection{Exploratory data analysis}\label{EDA}

\includegraphics{Proposal_EDA_files/figure-latex/unnamed-chunk-14-1.pdf}

\includegraphics{Proposal_EDA_files/figure-latex/unnamed-chunk-15-1.pdf}

\subsection{Salary model}\label{SM}

\begin{verbatim}
## 
## Call:
## lm(formula = sqrt(SALARY_MILLIONS) ~ POINTS + AGE + DRB, data = NBA_stat)
## 
## Residuals:
##      Min       1Q   Median       3Q      Max 
## -2.16618 -0.55411  0.00095  0.49505  2.48650 
## 
## Coefficients:
##             Estimate Std. Error t value Pr(>|t|)    
## (Intercept) -1.35822    0.32635  -4.162 4.50e-05 ***
## POINTS       0.09857    0.01026   9.607  < 2e-16 ***
## AGE          0.09753    0.01178   8.279 1.12e-14 ***
## DRB          0.10122    0.03578   2.829  0.00509 ** 
## ---
## Signif. codes:  0 '***' 0.001 '**' 0.01 '*' 0.05 '.' 0.1 ' ' 1
## 
## Residual standard error: 0.7832 on 224 degrees of freedom
## Multiple R-squared:  0.5669, Adjusted R-squared:  0.5611 
## F-statistic: 97.72 on 3 and 224 DF,  p-value: < 2.2e-16
\end{verbatim}

\subsection{Logistic model}\label{LM}

\begin{verbatim}
## 
## Call:
## glm(formula = eli ~ POINTS + AGE + DRB, family = binomial(link = "logit"), 
##     data = NBA_eli)
## 
## Deviance Residuals: 
##     Min       1Q   Median       3Q      Max  
## -2.0365  -0.5404  -0.2321   0.2954   2.7786  
## 
## Coefficients:
##             Estimate Std. Error z value Pr(>|z|)    
## (Intercept) -9.38966    1.71713  -5.468 4.55e-08 ***
## POINTS       0.23350    0.04254   5.488 4.05e-08 ***
## AGE          0.12336    0.05024   2.455   0.0141 *  
## DRB          0.64795    0.15119   4.286 1.82e-05 ***
## ---
## Signif. codes:  0 '***' 0.001 '**' 0.01 '*' 0.05 '.' 0.1 ' ' 1
## 
## (Dispersion parameter for binomial family taken to be 1)
## 
##     Null deviance: 272.30  on 224  degrees of freedom
## Residual deviance: 148.69  on 221  degrees of freedom
## AIC: 156.69
## 
## Number of Fisher Scoring iterations: 6
\end{verbatim}

\subsection{Team Valuation: variable selection}\label{VSR}

\begin{verbatim}
## 
## Call:
## lm(formula = lg_TEAM_EVALUATION ~ AST + TRB + STL + BLK + lg_GDP + 
##     lg_TWITTER_RETWEET_COUNT + lg_TICKET_PRICE, data = subset(teammat, 
##     select = c(-TEAM, -GDP, -GDP_POS, -TEAM_EVALUATION, -PAGEVIEWS, 
##         -TWITTER_FAVORITE_COUNT, -TWITTER_RETWEET_COUNT, -SALARY_MILLIONS, 
##         -TICKET_PRICE)))
## 
## Coefficients:
##              (Intercept)                       AST  
##                  1.18765                  -0.02764  
##                      TRB                       STL  
##                 -0.04159                   0.22708  
##                      BLK                    lg_GDP  
##                  0.12958                   0.29129  
## lg_TWITTER_RETWEET_COUNT           lg_TICKET_PRICE  
##                  0.23663                   0.24447
\end{verbatim}

\subsection{K-fold CV plot}\label{KCV}

\includegraphics{Proposal_EDA_files/figure-latex/unnamed-chunk-19-1.pdf}

\subsection{Scatter plot matrix of typical variables}\label{SPM}

\includegraphics{Proposal_EDA_files/figure-latex/unnamed-chunk-20-1.pdf}

\subsection{Variable selection plot with BIC criterion}\label{BIC2}

\includegraphics{Proposal_EDA_files/figure-latex/unnamed-chunk-21-1.pdf}


\end{document}
